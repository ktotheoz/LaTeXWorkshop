
% Dies gibt unsere Dokumentenklasse an
\documentclass[12pt]{article}

% Wir wollen das ganze auf Deutsch machen, andere Sprachen können aber natürlich auch verwendet werden
% Für die ganz spezielle Abgabe gibt es auch elbisch: http://tolklang.quettar.org/fonts/
\usepackage[ngerman]{babel}

% Falls utf8 nicht funktioniert (was es eigentlich definitiv tun sollte) kann man auch die folgenden Optionen verwenden:
%  Windows: ansinew
%  Linux: latin1
%  Max: applemac
\usepackage[utf8]{inputenc} 
\usepackage[T1]{fontenc} % Sollte immer benutzt werden um Probleme bei der Worttrennung zu vermeiden
\usepackage[a4paper]{geometry}

\usepackage{hyperref}
\usepackage{apacite}



% Das benutzen wir nur wenn wir eine 'einfache' Titelseite benutzen, anderenfalls löschen wir diese 3 Zeilen
% und fügen unsere etwas komplexere Titelseite direkt an der Stelle von '\maketitle' ein
\title{Seminarausarbeitung}
\author {Muster Student}
\date{ \today}






% Hier beginn unser Dokument
\begin{document}

% Hier können wir unseren Titel hinzufügen
\maketitle
\newpage

% Hier können wir ein Inhaltsverzeichnis einfügen
\tableofcontents

\newpage

\section{Mein erstes Kapitel}

\subsection{Mein erstes Unterkapitel}

Und ein bisschen Text darf auch nicht fehlen \cite{macklin2013position}

\section{Mein nächstes Kapitel}


% Mein Literaturverzeichnis
\newpage
\bibliographystyle{apacite}
\bibliography{meineLiteratur}

\end{document}
