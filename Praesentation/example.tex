\documentclass{beamer}
\usetheme{Singapore}
\usecolortheme{default}
\useinnertheme{circles}
\useoutertheme{default}

\usepackage[ngerman]{babel}
\usepackage[utf8]{inputenc}
\usepackage{xcolor}
\usepackage{verbatim}
\usepackage{amsmath}
\usepackage{mathtools}
\usepackage{standalone}
\usepackage{graphicx}
\usepackage{hyperref}


\begin{document}


\title{Beamer Beispiel}
\author{Muster Student}
\institute{Mein Institut}
\date{\today}

\begin{frame}
	\titlepage
\end{frame}


% Präsentationen können auch strukturiert werden
\section{Einf"uhrung}

\begin{frame}
	\frametitle{Hier der Titel der Folie}

	Und hier k"onnen wir nun einfach den Text einf"ugen.
\end{frame}

\begin{frame}
	\frametitle{Die zweite Folie}


	\begin{block}{Hier ein kleiner Block}
		Mit ein bisschen Inhalt.
	\end{block}
\end{frame}

\begin{frame}
	\frametitle{Und die dritte Folie}

	\begin{block}{Mit einer Overlay Liste}
		\begin{itemize}[<+->]
			\item Das erste Element
			\item Das zweite Element
		\end{itemize}
	\end{block}
	\pause
	\begin{block}{Nach der Pause}
		Eine kleine Demonstration des Pause Commandos.
	\end{block}
\end{frame}

\end{document}
