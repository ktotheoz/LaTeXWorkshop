\section{Beamer}



\subsection{Beamer Klasse}

\begin{frame}[<+->]
  \frametitle{Einleitung}

  \begin{block}{Vorteile}
  \begin{itemize}
    \item Komplexe Formeln stellen keinerlei Probleme dar
    \item Verlinkte Navigationsstrukturen
    \item Mehrere Versionen (Handout, Artikel, etc.)
    \item Erzeugung von PDF Dateien (Gut für Plattformunabgängigkeit)
  \end{itemize}
  \end{block}
\end{frame}


% Themes
\subsection{Themes}

\begin{frame}
  \frametitle{Themes}

  \begin{block}{usetheme}
    Um ein spezielles Theme auszuwählen braucht meein im Kopf der Datei eine weitere Anweisung \lstinline{usetheme{NAME}}.
  \end{block}

  \begin{block}{Standard Themen}
  \centering
	\begin{tabular}{c|c|c|c}
		Antibes & Bergen & Berkeley & Berlin \\\hline
		\textbf{Boadilla} & Copenhagen & Darmstadt & Dresden \\\hline
		Frankfurt & Goettingen & Hannover & Ilmenau \\\hline
		Juanlespins & Madrid & Malmoe & Marburg \\\hline
		Montpellier & Paloalto & Pittsburgh & Rochester \\\hline
		Singapore & Warsaw & &
	\end{tabular}
  \end{block}
\end{frame}

\begin{frame}
  \frametitle{Farbenschemas}

  \begin{block}{usecolortheme}
    Um ein spezielles Farbschema auszuwählen braucht meein im Kopf der Datei eine weitere Anweisung \lstinline{usecolortheme{NAME}}.
  \end{block}

  \begin{block}{Teils verschiedene Farbschemas}
  \centering
  \begin{tabular}{c|c|c|c}
  albatross & beaver & beetle & crane \\\hline
  \textbf{default} & dolphin & dove & fly \\\hline
  lily & orchid & rose & seagull \\\hline
  seahorse & sidebartab & structure & whale \\\hline
  wolverine & &
  \end{tabular}
  \end{block}
\end{frame}


% Folien erzeugen
\subsection{Folien erzeugen}

\begin{frame}[fragile]
  \frametitle{Grundgerüst}
  \begin{block}{Neue \textit{documentclass}}
    \begin{itemize}[<+->]
      \item Wir brauchen eine andere \textit{documentclass}.
      \item Bisher haben wir \textit{article} verwendet
      \item Für Präsentationen verwenden wir \textit{beamer}
    \end{itemize}
  \end{block}
  \pause
  \begin{block}{Neue Folie}
  \begin{verbatim}
    \begin{frame}
    \end{frame}
  \end{verbatim}
  \end{block}
\end{frame}

\begin{frame}[fragile]
  \frametitle{Einfache Folien}
    \begin{verbatim}

  \frametitle{Einfache Folien}
  \begin{itemize}
    \item Item 1
    \item Item 2
    \item Item 3
  \end{itemize}

    \end{verbatim}
    Ergebnis auf der nächsten Folie:
\end{frame}
\begin{frame}
  \frametitle{Einfache Folien}
  \begin{itemize}
    \item Item 1
    \item Item 2
    \item Item 3
  \end{itemize}
\end{frame}
% Langsam einblendende Listen
\begin{frame}[<+->]
  \frametitle{Einblendende Liste}
  \begin{itemize}
    \item Dieser Typ Folien
    \item ist eventuell aus anderen
    \item Veranstaltungen bekannt.
  \end{itemize}
\end{frame}
\begin{frame}[fragile]
  \frametitle{Einblendende Liste}
  \begin{block}{Code}
    \begin{verbatim}
begin{frame}[<+->]
  \begin{itemize}
    \item Dieser Typ Folien
    \item ist eventuell aus anderen
    \item Veranstaltungen bekannt.
  \end{itemize}
end{frame}
  \end{verbatim}
  \end{block}
\end{frame}
\begin{frame}[fragile]
  \frametitle{Explizite Overlays}

    \begin{overlayarea}{\textwidth}{4\baselineskip}
  \begin{block}{Explizite Spezifikationen}
      \visible<1>{Erscheint nur auf der ersten Folie} \\
      {\color<1-3>{green}{Dieser Text ist auf den Folien 1 bis 3 grün.} }\\
      \alert<3->{Ab Folie 3 erscheint dieser Text rot} \\
      \only<-4>{Dieser Text ist auf allen Folien, bis auf Folie 4} 
      \textbf<1, 3-4>{Dieser Text ist nur auf den Folien 1 und 3 bis 4 fett.} \\
      \alt<5>{Wir haben es fast geschafft ...}{Sind wir bald am Ende??} \\
  \end{block}
    \end{overlayarea}
\end{frame}
\begin{frame}[fragile]
  \frametitle{Explizite Overlays}

  \begin{block}{Code}
    \begin{verbatim}
    \begin{overlayarea}{\textwidth}{4\baselineskip}
  \begin{block}{Explizite Spezifikationen}
      \visible<1>{Erscheint nur auf der ersten Folie} \\
      {\color<1-3>{green}{Dieser Text ist auf den Folien 1 bis 3 grün.} }\\
      \alert<3->{Ab Folie 3 erscheint dieser Text rot} \\
      \only<-4>{Dieser Text ist auf allen Folien, bis auf Folie 4} 
      \textbf<1, 3-4>{Dieser Text ist nur auf den Folien 1 und 3 bis 4 fett.} \\
      \alt<5>{Wir haben es fast geschafft ...}{Sind wir bald am Ende??} \\
  \end{block}
    \end{overlayarea}
    \end{verbatim}
  \end{block}
\end{frame}

% Boxen
\begin{frame}
  \frametitle{Boxen}
  \begin{block}{Strukturen}
    \begin{itemize}[<+->]
      \item Mit Blöcken kann man etwas Struktur reinbringen
      \item Es gibt unterschiedliche vordefinierte Varianten
      \item Nun ein paar Beispiele was es denn so gibt.
    \end{itemize}
  \end{block}
  \pause
  \begin{block}{Meine Aufzählung}
    \begin{itemize}
      \item Example item 1
      \item Example item 2
      \item Example item 3
    \end{itemize}
  \end{block}
\end{frame}
\begin{frame}[fragile]
  \begin{verbatim}
  \begin{block}{Meine Aufzählung}
    \begin{itemize}
      \item Example item 1
      \item Example item 2
      \item Example item 3
    \end{itemize}
  \end{block}
  \end{verbatim}
\end{frame}
\begin{frame}
  \frametitle{Boxen 2}
  \begin{block}{Eine Definition}
  \begin{description}
    \item[${G_3}'$:] Die Menge R ist ausdr"uckbar.
  \end{description}
  \end{block}
\end{frame}
\begin{frame}[fragile]
  \begin{verbatim}
  \begin{block}{Eine Definition}
    \begin{description}
      \item[${G_3}'$:] Die Menge R ist ausdr"uckbar.
    \end{description}
  \end{block
  \end{verbatim}
\end{frame}
\begin{frame}
  \frametitle{Boxen 3}
  \begin{proof}
    Beweis
  \end{proof}
  \begin{definition}
    Definition
  \end{definition}
  \begin{example}
    Beispiel
  \end{example}
\end{frame}
\begin{frame}[fragile]
  \frametitle{Boxen 3}
  \begin{block}{Code}
    \begin{verbatim}
\begin{proof}
  Beweis
\end{proof}
\begin{definition}
  Definition
\end{definition}
\begin{example}
  Beispiel
\end{example}
    \end{verbatim}
  \end{block}
\end{frame}
